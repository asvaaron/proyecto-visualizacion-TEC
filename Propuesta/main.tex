%%%%%%%%%%%%%%%%%%%%%%%%%%%%%%%%%%%%%%%%%
% Lachaise Assignment
% LaTeX Template
% Version 1.0 (26/6/2018)
%
% This template originates from:
% http://www.LaTeXTemplates.com
%
% Authors:
% Marion Lachaise & François Févotte
% Vel (vel@LaTeXTemplates.com)
%
% License:
% CC BY-NC-SA 3.0 (http://creativecommons.org/licenses/by-nc-sa/3.0/)
% 
%%%%%%%%%%%%%%%%%%%%%%%%%%%%%%%%%%%%%%%%%

%----------------------------------------------------------------------------------------
%	PACKAGES AND OTHER DOCUMENT CONFIGURATIONS
%----------------------------------------------------------------------------------------


%Bibliograpy

\documentclass{article}

%%%%%%%%%%%%%%%%%%%%%%%%%%%%%%%%%%%%%%%%%
% Lachaise Assignment
% Structure Specification File
% Version 1.0 (26/6/2018)
%
% This template originates from:
% http://www.LaTeXTemplates.com
%
% Authors:
% Marion Lachaise & François Févotte
% Vel (vel@LaTeXTemplates.com)
%
% License:
% CC BY-NC-SA 3.0 (http://creativecommons.org/licenses/by-nc-sa/3.0/)
% 
%%%%%%%%%%%%%%%%%%%%%%%%%%%%%%%%%%%%%%%%%

%----------------------------------------------------------------------------------------
%	PACKAGES AND OTHER DOCUMENT CONFIGURATIONS
%----------------------------------------------------------------------------------------

\usepackage{amsmath,amsfonts,stmaryrd,amssymb} % Math packages

\usepackage{enumerate} % Custom item numbers for enumerations

\usepackage[ruled]{algorithm2e} % Algorithms

\usepackage[framemethod=tikz]{mdframed} % Allows defining custom boxed/framed environments

\usepackage{listings} % File listings, with syntax highlighting
\lstset{
	basicstyle=\ttfamily, % Typeset listings in monospace font
}


%Spanish languaje 
\usepackage[spanish]{babel}
%Bibliograpy
\usepackage{cite}
%Paragraph Space
\setlength{\parskip}{1em}
%Float
\usepackage{float}

% Footer and header styles 
\usepackage{fancyhdr}
\thispagestyle{plain}
%\pagestyle{fancy}
%\fancyhf{}
%\fancyhead[LE,RO]{Tarea 3}
%\fancyhead[RE,LO]{\leftmark}
%\fancyfoot[CE,CO]{\thepage }
%\fancyfoot[LE,RO]{
%	\begin{figure}[H]
%		\flushright
%		\includegraphics[width=4cm, height=2cm]{./img/tec.png}
%	\end{figure}
%}

\renewcommand{\baselinestretch}{1.5}

%----------------------------------------------------------------------------------------
%	DOCUMENT MARGINS
%----------------------------------------------------------------------------------------

\usepackage{geometry} % Required for adjusting page dimensions and margins

\geometry{
	paper=a4paper, % Paper size, change to letterpaper for US letter size
	top=2.5cm, % Top margin
	bottom=3cm, % Bottom margin
	left=2.5cm, % Left margin
	right=2.5cm, % Right margin
	headheight=14pt, % Header height
	footskip=1.5cm, % Space from the bottom margin to the baseline of the footer
	headsep=1.2cm, % Space from the top margin to the baseline of the header
	%showframe, % Uncomment to show how the type block is set on the page
}

%----------------------------------------------------------------------------------------
%	FONTS
%----------------------------------------------------------------------------------------

\usepackage[utf8]{inputenc} % Required for inputting international characters
\usepackage[T1]{fontenc} % Output font encoding for international characters

\usepackage{XCharter} % Use the XCharter fonts

%----------------------------------------------------------------------------------------
%	COMMAND LINE ENVIRONMENT
%----------------------------------------------------------------------------------------

% Usage:
% \begin{commandline}
%	\begin{verbatim}
%		$ ls
%		
%		Applications	Desktop	...
%	\end{verbatim}
% \end{commandline}

\mdfdefinestyle{commandline}{
	leftmargin=10pt,
	rightmargin=10pt,
	innerleftmargin=15pt,
	middlelinecolor=black!50!white,
	middlelinewidth=2pt,
	frametitlerule=false,
	backgroundcolor=black!5!white,
	frametitle={Command Line},
	frametitlefont={\normalfont\sffamily\color{white}\hspace{-1em}},
	frametitlebackgroundcolor=black!50!white,
	nobreak,
}

% Define a custom environment for command-line snapshots
\newenvironment{commandline}{
	\medskip
	\begin{mdframed}[style=commandline]
}{
	\end{mdframed}
	\medskip
}

%----------------------------------------------------------------------------------------
%	FILE CONTENTS ENVIRONMENT
%----------------------------------------------------------------------------------------

% Usage:
% \begin{file}[optional filename, defaults to "File"]
%	File contents, for example, with a listings environment
% \end{file}

\mdfdefinestyle{file}{
	innertopmargin=1.6\baselineskip,
	innerbottommargin=0.8\baselineskip,
	topline=false, bottomline=false,
	leftline=false, rightline=false,
	leftmargin=2cm,
	rightmargin=2cm,
	singleextra={%
		\draw[fill=black!10!white](P)++(0,-1.2em)rectangle(P-|O);
		\node[anchor=north west]
		at(P-|O){\ttfamily\mdfilename};
		%
		\def\l{3em}
		\draw(O-|P)++(-\l,0)--++(\l,\l)--(P)--(P-|O)--(O)--cycle;
		\draw(O-|P)++(-\l,0)--++(0,\l)--++(\l,0);
	},
	nobreak,
}

% Define a custom environment for file contents
\newenvironment{file}[1][File]{ % Set the default filename to "File"
	\medskip
	\newcommand{\mdfilename}{#1}
	\begin{mdframed}[style=file]
}{
	\end{mdframed}
	\medskip
}

%----------------------------------------------------------------------------------------
%	NUMBERED QUESTIONS ENVIRONMENT
%----------------------------------------------------------------------------------------

% Usage:
% \begin{question}[optional title]
%	Question contents
% \end{question}

\mdfdefinestyle{question}{
	innertopmargin=1.2\baselineskip,
	innerbottommargin=0.8\baselineskip,
	roundcorner=5pt,
	nobreak,
	singleextra={%
		\draw(P-|O)node[xshift=1em,anchor=west,fill=white,draw,rounded corners=5pt]{%
		Question \theQuestion\questionTitle};
	},
}

\newcounter{Question} % Stores the current question number that gets iterated with each new question

% Define a custom environment for numbered questions
\newenvironment{question}[1][\unskip]{
	\bigskip
	\stepcounter{Question}
	\newcommand{\questionTitle}{~#1}
	\begin{mdframed}[style=question]
}{
	\end{mdframed}
	\medskip
}

%----------------------------------------------------------------------------------------
%	WARNING TEXT ENVIRONMENT
%----------------------------------------------------------------------------------------

% Usage:
% \begin{warn}[optional title, defaults to "Warning:"]
%	Contents
% \end{warn}

\mdfdefinestyle{warning}{
	topline=false, bottomline=false,
	leftline=false, rightline=false,
	nobreak,
	singleextra={%
		\draw(P-|O)++(-0.5em,0)node(tmp1){};
		\draw(P-|O)++(0.5em,0)node(tmp2){};
		\fill[black,rotate around={45:(P-|O)}](tmp1)rectangle(tmp2);
		\node at(P-|O){\color{white}\scriptsize\bf !};
		\draw[very thick](P-|O)++(0,-1em)--(O);%--(O-|P);
	}
}

% Define a custom environment for warning text
\newenvironment{warn}[1][Warning:]{ % Set the default warning to "Warning:"
	\medskip
	\begin{mdframed}[style=warning]
		\noindent{\textbf{#1}}
}{
	\end{mdframed}
}

%----------------------------------------------------------------------------------------
%	INFORMATION ENVIRONMENT
%----------------------------------------------------------------------------------------

% Usage:
% \begin{info}[optional title, defaults to "Info:"]
% 	contents
% 	\end{info}

\mdfdefinestyle{info}{%
	topline=false, bottomline=false,
	leftline=false, rightline=false,
	nobreak,
	singleextra={%
		\fill[black](P-|O)circle[radius=0.4em];
		\node at(P-|O){\color{white}\scriptsize\bf i};
		\draw[very thick](P-|O)++(0,-0.8em)--(O);%--(O-|P);
	}
}

% Define a custom environment for information
\newenvironment{info}[1][Info:]{ % Set the default title to "Info:"
	\medskip
	\begin{mdframed}[style=info]
		\noindent{\textbf{#1}}
}{
	\end{mdframed}
}

 % Include the file specifying the document structure and custom commands

%----------------------------------------------------------------------------------------
%	ASSIGNMENT INFORMATION
%----------------------------------------------------------------------------------------

\title{Proyecto: Nombre Proyecto} % Title of the assignment

\author{Geykel Hodgson Chavaría\\ 
Aaron Sibaja Villalobos  
} % Author name and email address

\date{Instituto  Tecnológico de Costa Rica -- \today} % University, school and/or department name(s) and a date

%----------------------------------------------------------------------------------------

\begin{document}

\maketitle % Print the title

%----------------------------------------------------------------------------------------
%	INTRODUCTION
%----------------------------------------------------------------------------------------

\section{Definición de problema}
Se sabe que existe una relación entre el consumo eléctrico y el crecimiento económico de un país, lo que no esta claro es si las energías limpias influyen de alguna forma en el crecimiento del PIB. El propósito de esta visualización es identificar las relaciones en el crecimiento económico al comparar países que apuesta por energía renovables contra los que utilizan derribados del petroleo en América Latina.

\section{Objetivos}
\subsection{General}
Identificar las relaciones del crecimiento económico y el uso de energías limpias en América Latina.
\subsection{Específicos }
\begin{itemize}
    \item Identificar los tipos de energía más usados en América Latina.
    % \item Determinar el crecimiento del PIB de los países de América Latina.
    \item Conocer la cobertura de acceso a la electricidad por país en América Latina.
    \item Comparar crecimiento económico entre países que apuestan por energías limpias contra países que utilizan combustibles fósiles en América Latina.
    \item Evidenciar causalidad entre el consumo de fuentes renovables y el incremento del PIB.
\end{itemize}

\section{Contexto}

Existe un cantidad significante de artículos que han mostrado cómo el crecimiento  de la actividad económica de un país se encuentra relacionado con el uso de la electricidad. Esto evidencia un crecimiento en la población y en la generación de bienes y servicios \cite{chen_relationship_2007}. 



\section{Metodología }
Para la realización de este trabajo se decidió utilizar la metodología propuesta por Benjamín Fry \cite{fry2008visualizing}, este método consta de siete etapas:
\begin{itemize}
    \item Recolectar datos.
    Esta etapa consiste en buscar y obtener el conjunto de datos que vamos a utilizar, lo principal es conseguir fuentes de datos confiables, provenientes de otros estudios académicos o algún organismo oficial. En esta fase también se define como los datos recolectados van a ser accedidos por los usuarios.
    \item Estructuración de los datos.
    Una vez se tienen datos confiables, hay que ordenarlos y categorizarlos para que estos adquieran una estructura bien definida. Si lo datos no están estructurados correctamente se dificulta el análisis y la visualización de los mismos.
    \item Filtrar.
    En este paso se debe decidir desde el punto de vista de la visualización cuales datos son útiles y aportan información relevante. Los datos que no se consideren relevantes son filtrados y no se toman en cuenta para la representación.
    \item Extraer.
    Se procede a convertir, los datos relevantes, en variables que denoten los valores o cantidades que necesitamos analizar y mostrar a través de la visualización. Esto se logra al aplicar métodos estadísticos o de minería de datos que permitan colocar los datos en un contexto matemático.
    \item Representar.
    Se escoge un modelo visual básico (Gráfico de barras, arboles, listas, etc), apropiado para una visualización clara y concisa de las variables creadas en el paso anterior.
    \item Refinar.
    Trabajar de forma iterativa sobre el modelo visual básico con el fin de mejorarlo y que sea capaz de transmitir la información de forma clara, además debe ser atractivo y cautivador para el usuario.  
    \item Interactuar.
    Agregar elementos que permitan al usuario la manipulación y el control de las características que son visibles. Los usuarios deben poder seleccionar diferentes rangos de datos e intervalos de tiempo que le permitan analizar la información presentada desde múltiples puntos de vista.  
\end{itemize}

\section{Alcance Esperado}
*** Exploratorio
- Examinar un tema poco estudiado (se ha estudiado mucho la relación entre electricidad y crecimiento económico pero poco o nada en relación al tipo de energía utilizado)
- Indaga desde una perspectiva innovadora (la perspectiva de energías renovables o limpias)
- Ayuda a identificar conceptos promisorios (Identificar causalidad en el uso de energías limpias y crecimiento económico)

-Prepara el terreno para nuevos estudios, que otros estudios podrían derivarse?

el crecimiento economico se va medir desde el punto de vista del PIB.
se dividen los tipos de energia entre derivados del petroleo y energias renovables (cuales energias renovables toma encuenta el banco mundial?)
muestra el porcentaje de acceso a la electricidad en cada pais ? %puede no ser relevante 

\section{Recursos requeridos}

Se va a utilizar los datos recolectados por el banco mundial los cuales son presentados mediante su página web oficial \cite{noauthor_indicadores_nodate}. Este portal contiene 76 base de datos en total, las cuales abarcan 264 países. Cada base de datos está compuesta por series, en la base de datos principal \textit{indicadores del desarrollo mundial} se pueden encontrar 1429, las series corresponden a indicadores y se encuentran etiquetadas con su respectiva descripción y con diferente medición numérica.

La página web permite realizar filtros de forma que se puedan tener varias regiones del globo. Además admite seleccionar el tiempo en los datos de forma que cada indicador es anual para cada país estos se encuentran desde 1966 hasta el 2019. 

Es posible extraer los datos en formato de Excel, CSV o en TSV. La descarga es un archivo comprimido que contiene dos archivos, uno con la data con el formato seleccionado y otro  archivo con metadata detallada sobre cada uno de los indicadores cómo por ejemplo: fuente, descripción larga y corta, licencia y periodicidad. 


\section{Lista de riesgos y estrategias de mitigación}

\section{Cronograma}

\section*{Introduction} % Unnumbered section

\begin{info} % Information block
	This is an interesting piece of information, to which the reader should pay special attention. Fusce varius orci ac magna dapibus porttitor. In tempor leo a neque bibendum sollicitudin. Nulla pretium fermentum nisi, eget sodales magna facilisis eu. Praesent aliquet nulla ut bibendum lacinia. Donec vel mauris vulputate, commodo ligula ut, egestas orci. Suspendisse commodo odio sed hendrerit lobortis. Donec finibus eros erat, vel ornare enim mattis et.
\end{info}

% Numbered question, with subquestions in an enumerate environment
\begin{question}
	Quisque ullamcorper placerat ipsum. Cras nibh. Morbi vel justo vitae lacus tincidunt ultrices. Lorem ipsum dolor sit amet, consectetuer adipiscing elit.

	% Subquestions numbered with letters
	\begin{enumerate}[(a)]
		\item Do this.
		\item Do that.
		\item Do something else.
	\end{enumerate}
\end{question}
	
%------------------------------------------------


\begin{center}
	\begin{minipage}{0.5\linewidth} % Adjust the minipage width to accomodate for the length of algorithm lines
		\begin{algorithm}[H]
			\KwIn{$(a, b)$, two floating-point numbers}  % Algorithm inputs
			\KwResult{$(c, d)$, such that $a+b = c + d$} % Algorithm outputs/results
			\medskip
			\If{$\vert b\vert > \vert a\vert$}{
				exchange $a$ and $b$ \;
			}
			$c \leftarrow a + b$ \;
			$z \leftarrow c - a$ \;
			$d \leftarrow b - z$ \;
			{\bf return} $(c,d)$ \;
			\caption{\texttt{FastTwoSum}} % Algorithm name
			\label{alg:fastTwoSum}   % optional label to refer to
		\end{algorithm}
	\end{minipage}
\end{center}

% Numbered question, with an optional title
\begin{question}[\itshape (with optional title)]
	In congue risus leo, in gravida enim viverra id. Donec eros mauris, bibendum vel dui at, tempor commodo augue. In vel lobortis lacus. Nam ornare ullamcorper mauris vel molestie. Maecenas vehicula ornare turpis, vitae fringilla orci consectetur vel. Nam pulvinar justo nec neque egestas tristique. Donec ac dolor at libero congue varius sed vitae lectus. Donec et tristique nulla, sit amet scelerisque orci. Maecenas a vestibulum lectus, vitae gravida nulla. Proin eget volutpat orci. Morbi eu aliquet turpis. Vivamus molestie urna quis tempor tristique. Proin hendrerit sem nec tempor sollicitudin.
\end{question}


% File contents
\begin{file}[hello.py]
\begin{lstlisting}[language=Python]
#! /usr/bin/python

import sys
sys.stdout.write("Hello World!\n")
\end{lstlisting}
\end{file}


% Command-line "screenshot"
\begin{commandline}
	\begin{verbatim}
		$ chmod +x hello.py
		$ ./hello.py

		Hello World!
	\end{verbatim}
\end{commandline}

% Warning text, with a custom title
\begin{warn}[Notice:]
  In congue risus leo, in gravida enim viverra id. Donec eros mauris, bibendum vel dui at, tempor commodo augue. In vel lobortis lacus. Nam ornare ullamcorper mauris vel molestie. Maecenas vehicula ornare turpis, vitae fringilla orci consectetur vel. Nam pulvinar justo nec neque egestas tristique. Donec ac dolor at libero congue varius sed vitae lectus. Donec et tristique nulla, sit amet scelerisque orci. Maecenas a vestibulum lectus, vitae gravida nulla. Proin eget volutpat orci. Morbi eu aliquet turpis. Vivamus molestie urna quis tempor tristique. Proin hendrerit sem nec tempor sollicitudin.
\end{warn}

%----------------------------------------------------------------------------------------

\bibliography{proyecto_visualizacion}{}
\bibliographystyle{acm}

\end{document}

