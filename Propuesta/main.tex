%%%%%%%%%%%%%%%%%%%%%%%%%%%%%%%%%%%%%%%%%
% Lachaise Assignment
% LaTeX Template
% Version 1.0 (26/6/2018)
%
% This template originates from:
% http://www.LaTeXTemplates.com
%
% Authors:
% Marion Lachaise & François Févotte
% Vel (vel@LaTeXTemplates.com)
%
% License:
% CC BY-NC-SA 3.0 (http://creativecommons.org/licenses/by-nc-sa/3.0/)
% 
%%%%%%%%%%%%%%%%%%%%%%%%%%%%%%%%%%%%%%%%%

%----------------------------------------------------------------------------------------
%	PACKAGES AND OTHER DOCUMENT CONFIGURATIONS
%----------------------------------------------------------------------------------------


%Bibliograpy

\documentclass{article}

\input{structure.tex} % Include the file specifying the document structure and custom commands


%----------------------------------------------------------------------------------------
%	ASSIGNMENT INFORMATION
%----------------------------------------------------------------------------------------

\title{Proyecto: Crecimiento económico y su relación con energías limpias en la región latinoamericana} % Title of the assignment

\author{Geykel Hodgson Chavaría\\ 
Aaron Sibaja Villalobos  
} % Author name and email address

\date{Instituto  Tecnológico de Costa Rica -- \today} % University, school and/or department name(s) and a date

%----------------------------------------------------------------------------------------

\begin{document}

%Rename table name
\renewcommand{\listtablename}{Índice de tablas}
\renewcommand{\tablename}{Tabla} 

\maketitle % Print the title

%----------------------------------------------------------------------------------------
%	INTRODUCTION
%----------------------------------------------------------------------------------------

\section{Definición de problema}
Se sabe que existe una relación entre el consumo eléctrico y el crecimiento económico de un país, lo que no esta claro es si las energías limpias influyen de alguna forma en el crecimiento del PIB. El propósito de esta visualización es identificar las relaciones en el crecimiento económico al comparar países que apuesta por energía renovables contra los que utilizan derribados del petroleo en América Latina.

\section{Objetivos}
\subsection{General}
Identificar las relaciones del crecimiento económico y el uso de energías limpias en América Latina.
\subsection{Específicos }
\begin{itemize}
    \item Identificar los tipos de energía más usados en América Latina.
    % \item Determinar el crecimiento del PIB de los países de América Latina.
    \item Conocer la cobertura de acceso a la electricidad por país en América Latina.
    \item Comparar crecimiento económico entre países que apuestan por energías limpias contra países que utilizan combustibles fósiles en América Latina.
    \item Evidenciar causalidad entre el consumo de fuentes renovables y el incremento del PIB.
\end{itemize}

\section{Contexto}




La electricidad corresponde a un pilar fundamental para progreso de una nación.   Existe un cantidad significante de textos y artículos que han demostrado cómo el crecimiento  de la actividad económica de un país se encuentra relacionado con el uso de la electricidad. Esto evidencia un crecimiento en la población y en la generación de bienes y servicios \cite{chen_relationship_2007}. 

El PIB  corresponde a la suma del sector  industrial, agricultura y de servicios de un país, en otras palabras, el valor en el mercado de sus productos y servicios. El sector agricultura y de servicios son sectores poco dependientes de la electricidad sin embargo el sector industrial si depende mayormente de la electricidad. Esto da cómo que la relación entre el PIB y el consumo de electricidad en países que basan mayormente su economía en el sector industrial. Para el 2017, 20$\%$ del PIB en Estados Unidos estaba conformada por el sector industrial \cite{noauthor_why_2014}. 

Algunos países que pertenecen al OCDE (Organización para la Cooperación y el Desarrollo Económicos ) cómo por ejemplo: Canadá, Reino Unido, Suiza, USA, Alemania y Colombia tienen economías fuertemente influenciadas por  la manufactura, sin embargo continuamente se actualizan a tecnologías que hacen uso de un menor recurso eléctrico y son más eficientes \cite{noauthor_link_nodate}. Los paises que no forman parte del OCDE cómo China, India y Brazil, presentan economías en crecimiento generalmente basadas en sector manufactura. Estas economías en particular presentan tecnologías más antiguas que requieren un mayor consumo de energía para poder generar bienes. En el 2011 el total de uso de energía de los países que no pertenecen al OCDE sobrepasó a los miembros del OCDE. Esto evidencia que probablemente la mayor cantidad de consumo eléctrico en el futuro va a concentrar en los paises que no forman parte del OCDE y el camino que elijan para sus respectivas economías. 










\section{Metodología }
Para la realización de este trabajo se decidió utilizar la metodología propuesta por Benjamín Fry \cite{fry2008visualizing}, este método consta de siete etapas:
\begin{itemize}
    \item \textbf{Recolectar datos: } 
    esta etapa consiste en buscar y obtener el conjunto de datos que vamos a utilizar, lo principal es conseguir fuentes de datos confiables, provenientes de otros estudios académicos o algún organismo oficial. En esta fase también se define como los datos recolectados van a ser accedidos por los usuarios.
    \item \textbf{Estructuración de los datos: } 
    una vez que se tienen datos confiables, hay que ordenarlos y categorizarlos para que estos adquieran una estructura bien definida. Si lo datos no están estructurados correctamente se dificulta el análisis y la visualización de los mismos.
    \item \textbf{Filtrar: }
    en este paso se debe decidir desde el punto de vista de la visualización cuales datos son útiles y aportan información relevante. Los datos que no se consideren relevantes son filtrados y no se toman en cuenta para la representación.
    \item \textbf{Extraer: }
    se procede a convertir, los datos relevantes, en variables que denoten los valores o cantidades que necesitamos analizar y mostrar a través de la visualización. Esto se logra al aplicar métodos estadísticos o de minería de datos que permitan colocar los datos en un contexto matemático.
    \item \textbf{Representar:}
    se escoge un modelo visual básico (Gráfico de barras, arboles, listas, entre otros), apropiado para una visualización clara y concisa de las variables creadas en el paso anterior.
    \item \textbf{Refinar: }
    trabajar de forma iterativa sobre el modelo visual básico con el fin de mejorarlo y que sea capaz de transmitir la información de forma clara, además debe ser atractivo y cautivador para el usuario.  
    \item \textbf{Interactuar: }
    agregar elementos que permitan al usuario la manipulación y el control de las características que son visibles. Los usuarios deben poder seleccionar diferentes rangos de datos e intervalos de tiempo que le permitan analizar la información presentada desde múltiples puntos de vista.  
\end{itemize}

\section{Alcance Esperado}
*** Exploratorio
- Examinar un tema poco estudiado (se ha estudiado mucho la relación entre electricidad y crecimiento económico pero poco o nada en relación al tipo de energía utilizado)
- Indaga desde una perspectiva innovadora (la perspectiva de energías renovables o limpias)
- Ayuda a identificar conceptos promisorios (Identificar causalidad en el uso de energías limpias y crecimiento económico)

-Prepara el terreno para nuevos estudios, que otros estudios podrían derivarse?

el crecimiento economico se va medir desde el punto de vista del PIB.
se dividen los tipos de energia entre derivados del petroleo y energias renovables (cuales energias renovables toma encuenta el banco mundial?)
muestra el porcentaje de acceso a la electricidad en cada pais ? %puede no ser relevante 

\section{Recursos requeridos}

Se va a utilizar los datos recolectados por el banco mundial los cuales son presentados mediante su página web oficial \cite{noauthor_indicadores_nodate}. Este portal contiene 76 base de datos en total, las cuales abarcan 264 países. Cada base de datos está compuesta por series, en la base de datos principal \textit{indicadores del desarrollo mundial} se pueden encontrar 1429, las series corresponden a indicadores y se encuentran etiquetadas con su respectiva descripción y con diferente medición numérica.

La página web permite realizar filtros de forma que se puedan tener varias regiones del globo. Además admite seleccionar el tiempo en los datos de forma que cada indicador es anual para cada país estos se encuentran desde 1966 hasta el 2019. 

Es posible extraer los datos en formato de Excel, CSV o en TSV. La descarga es un archivo comprimido que contiene dos archivos, uno con la data con el formato seleccionado y otro  archivo con metadata detallada sobre cada uno de los indicadores cómo por ejemplo: fuente, descripción larga y corta, licencia y periodicidad. 


\section{Lista de riesgos y estrategias de mitigación}

%The below command generates a 4 X 5 risk matrix with ALARP regions for initial risk matrix.

\begin{table}[H]
\centering
\scriptsize
\caption{Lista de riesgos}
\begin{tabular}{|p{2cm}|p{2cm}|p{2cm}| p{2cm} |p{2cm}| p{2cm}|}
\hline \bf Frecuencia/Consecuencia & \bf Poco Probable & \bf Probable & \bf Ocasionalmente & \bf Probable & \bf Bastante Probable\\ [10pt]

\hline \bf 4-Catastrófico & \cellcolor{yellow!50} & \cellcolor{red!50} & \cellcolor{red!50} & \cellcolor{red!50} &\cellcolor{red!50} \\ [10pt]

\hline \bf 3-Critico &\cellcolor{green!50} & \cellcolor{yellow!50} & \cellcolor{yellow!50} & \cellcolor{red!50} &\cellcolor{red!50} \\ [10pt]

\hline \bf 2-Mayor & \cellcolor{green!50} & \cellcolor{green!50} & \cellcolor{yellow!50} &\cellcolor{yellow!50} &\cellcolor{red!50} \\ [10pt]

\hline \bf 1-Menor & \cellcolor{green!50} & \cellcolor{green!50} & \cellcolor{green!50} &\cellcolor{yellow!50} &\cellcolor{yellow!50} \\ [10pt]
\hline
\end{tabular} \\
\end{table}

 


%The command below provides the legend for the Risk Matrix

\begin{table}[H]
\centering
%\scriptsize
\caption{Legenda de color de la matriz de riesgo}
\begin{tabular}{|p{2cm}|p{10cm}|}
\hline \bf Color & \bf Legenda \\
\hline \cellcolor{red! 50} & Aceptable - Se requiere reducción de riesgo\\ [10pt]
\hline \cellcolor{yellow! 50} & Aceptable - Considere reducir el riesgo. \\[10pt]
\hline \cellcolor{green! 50} & Aceptable \\ [10pt]
\hline
\end{tabular}
\end{table}


\section{Cronograma}

\begin{figure}[H]
		\centering
		\includegraphics[width=18cm, height=14cm, keepaspectratio]{./img/cronograma_proyecto.png}
		\caption{Cronograma de trabajo }
		\label{fig:cronograma}
	\end{figure}


\section*{Introduction} % Unnumbered section

\begin{info} % Information block
	This is an interesting piece of information, to which the reader should pay special attention. Fusce varius orci ac magna dapibus porttitor. In tempor leo a neque bibendum sollicitudin. Nulla pretium fermentum nisi, eget sodales magna facilisis eu. Praesent aliquet nulla ut bibendum lacinia. Donec vel mauris vulputate, commodo ligula ut, egestas orci. Suspendisse commodo odio sed hendrerit lobortis. Donec finibus eros erat, vel ornare enim mattis et.
\end{info}

% Numbered question, with subquestions in an enumerate environment
\begin{question}
	Quisque ullamcorper placerat ipsum. Cras nibh. Morbi vel justo vitae lacus tincidunt ultrices. Lorem ipsum dolor sit amet, consectetuer adipiscing elit.

	% Subquestions numbered with letters
	\begin{enumerate}[(a)]
		\item Do this.
		\item Do that.
		\item Do something else.
	\end{enumerate}
\end{question}
	
%------------------------------------------------


\begin{center}
	\begin{minipage}{0.5\linewidth} % Adjust the minipage width to accomodate for the length of algorithm lines
		\begin{algorithm}[H]
			\KwIn{$(a, b)$, two floating-point numbers}  % Algorithm inputs
			\KwResult{$(c, d)$, such that $a+b = c + d$} % Algorithm outputs/results
			\medskip
			\If{$\vert b\vert > \vert a\vert$}{
				exchange $a$ and $b$ \;
			}
			$c \leftarrow a + b$ \;
			$z \leftarrow c - a$ \;
			$d \leftarrow b - z$ \;
			{\bf return} $(c,d)$ \;
			\caption{\texttt{FastTwoSum}} % Algorithm name
			\label{alg:fastTwoSum}   % optional label to refer to
		\end{algorithm}
	\end{minipage}
\end{center}

% Numbered question, with an optional title
\begin{question}[\itshape (with optional title)]
	In congue risus leo, in gravida enim viverra id. Donec eros mauris, bibendum vel dui at, tempor commodo augue. In vel lobortis lacus. Nam ornare ullamcorper mauris vel molestie. Maecenas vehicula ornare turpis, vitae fringilla orci consectetur vel. Nam pulvinar justo nec neque egestas tristique. Donec ac dolor at libero congue varius sed vitae lectus. Donec et tristique nulla, sit amet scelerisque orci. Maecenas a vestibulum lectus, vitae gravida nulla. Proin eget volutpat orci. Morbi eu aliquet turpis. Vivamus molestie urna quis tempor tristique. Proin hendrerit sem nec tempor sollicitudin.
\end{question}


% File contents
\begin{file}[hello.py]
\begin{lstlisting}[language=Python]
#! /usr/bin/python

import sys
sys.stdout.write("Hello World!\n")
\end{lstlisting}
\end{file}


% Command-line "screenshot"
\begin{commandline}
	\begin{verbatim}
		$ chmod +x hello.py
		$ ./hello.py

		Hello World!
	\end{verbatim}
\end{commandline}

% Warning text, with a custom title
\begin{warn}[Notice:]
  In congue risus leo, in gravida enim viverra id. Donec eros mauris, bibendum vel dui at, tempor commodo augue. In vel lobortis lacus. Nam ornare ullamcorper mauris vel molestie. Maecenas vehicula ornare turpis, vitae fringilla orci consectetur vel. Nam pulvinar justo nec neque egestas tristique. Donec ac dolor at libero congue varius sed vitae lectus. Donec et tristique nulla, sit amet scelerisque orci. Maecenas a vestibulum lectus, vitae gravida nulla. Proin eget volutpat orci. Morbi eu aliquet turpis. Vivamus molestie urna quis tempor tristique. Proin hendrerit sem nec tempor sollicitudin.
\end{warn}

%----------------------------------------------------------------------------------------

\bibliography{proyecto_visualizacion}{}
\bibliographystyle{acm}

\end{document}

