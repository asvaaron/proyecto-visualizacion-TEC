%%%%%%%%%%%%%%%%%%%%%%%%%%%%%%%%%%%%%%%%%
% Lachaise Assignment
% LaTeX Template
% Version 1.0 (26/6/2018)
%
% This template originates from:
% http://www.LaTeXTemplates.com
%
% Authors:
% Marion Lachaise & François Févotte
% Vel (vel@LaTeXTemplates.com)
%
% License:
% CC BY-NC-SA 3.0 (http://creativecommons.org/licenses/by-nc-sa/3.0/)
% 
%%%%%%%%%%%%%%%%%%%%%%%%%%%%%%%%%%%%%%%%%

%----------------------------------------------------------------------------------------
%	PACKAGES AND OTHER DOCUMENT CONFIGURATIONS
%----------------------------------------------------------------------------------------


%Bibliograpy

\documentclass{article}

\input{structure.tex} % Include the file specifying the document structure and custom commands

%----------------------------------------------------------------------------------------
%	ASSIGNMENT INFORMATION
%----------------------------------------------------------------------------------------

\title{Proyecto: Nombre Proyecto} % Title of the assignment

\author{Geykel Hodgson Chavaría\\ 
Aaron Sibaja Villalobos  
} % Author name and email address

\date{Instituto  Tecnológico de Costa Rica -- \today} % University, school and/or department name(s) and a date

%----------------------------------------------------------------------------------------

\begin{document}

\maketitle % Print the title

%----------------------------------------------------------------------------------------
%	INTRODUCTION
%----------------------------------------------------------------------------------------

\section{Definición de problema}
Se sabe que existe una relación entre el consumo eléctrico y el crecimiento económico de un país, lo que no esta claro es si las energías limpias influyen de alguna forma en el crecimiento del PIB. El propósito de esta visualización es identificar las relaciones en el crecimiento económico al comparar países que apuesta por energía renovables contra los que utilizan derribados del petroleo en América Latina.

\section{Objetivos}
\subsection{General}
Identificar las relaciones del crecimiento económico y el uso de energías limpias en América Latina.
\subsection{Específicos }
\begin{itemize}
    \item Identificar los tipos de energía más usados en América Latina.
    % \item Determinar el crecimiento del PIB de los países de América Latina.
    \item Conocer la cobertura de acceso a la electricidad por país en América Latina.
    \item Comparar crecimiento económico entre países que apuestan por energías limpias contra países que utilizan combustibles fósiles en América Latina.
    \item Evidenciar causalidad entre el consumo de fuentes renovables y el incremento del PIB.
\end{itemize}

\section{Contexto}

Existe un cantidad significante de artículos que han mostrado cómo el crecimiento  de la actividad económica de un país se encuentra relacionado con el uso de la electricidad. Esto evidencia un crecimiento en la población y en la generación de bienes y servicios \cite{chen_relationship_2007}. 



\section{Metodología }

\section{Alcance Esperado}

\section{Recursos requeridos}

Se va a utilizar los datos recolectados por el banco mundial los cuales son presentados mediante su página web oficial \cite{noauthor_indicadores_nodate}. Este portal contiene 76 base de datos en total, las cuales abarcan 264 países. Cada base de datos está compuesta por series, en la base de datos principal \textit{indicadores del desarrollo mundial} se pueden encontrar 1429, las series corresponden a indicadores y se encuentran etiquetadas con su respectiva descripción y con diferente medición numérica.

La página web permite realizar filtros de forma que se puedan tener varias regiones del globo. Además admite seleccionar el tiempo en los datos de forma que cada indicador es anual para cada país estos se encuentran desde 1966 hasta el 2019. 

Es posible extraer los datos en formato de Excel, CSV o en TSV. La descarga es un archivo comprimido que contiene dos archivos, uno con la data con el formato seleccionado y otro  archivo con metadata detallada sobre cada uno de los indicadores cómo por ejemplo: fuente, descripción larga y corta, licencia y periodicidad. 


\section{Lista de riesgos y estrategias de mitigación}

\section{Cronograma}

\section*{Introduction} % Unnumbered section

\begin{info} % Information block
	This is an interesting piece of information, to which the reader should pay special attention. Fusce varius orci ac magna dapibus porttitor. In tempor leo a neque bibendum sollicitudin. Nulla pretium fermentum nisi, eget sodales magna facilisis eu. Praesent aliquet nulla ut bibendum lacinia. Donec vel mauris vulputate, commodo ligula ut, egestas orci. Suspendisse commodo odio sed hendrerit lobortis. Donec finibus eros erat, vel ornare enim mattis et.
\end{info}

% Numbered question, with subquestions in an enumerate environment
\begin{question}
	Quisque ullamcorper placerat ipsum. Cras nibh. Morbi vel justo vitae lacus tincidunt ultrices. Lorem ipsum dolor sit amet, consectetuer adipiscing elit.

	% Subquestions numbered with letters
	\begin{enumerate}[(a)]
		\item Do this.
		\item Do that.
		\item Do something else.
	\end{enumerate}
\end{question}
	
%------------------------------------------------


\begin{center}
	\begin{minipage}{0.5\linewidth} % Adjust the minipage width to accomodate for the length of algorithm lines
		\begin{algorithm}[H]
			\KwIn{$(a, b)$, two floating-point numbers}  % Algorithm inputs
			\KwResult{$(c, d)$, such that $a+b = c + d$} % Algorithm outputs/results
			\medskip
			\If{$\vert b\vert > \vert a\vert$}{
				exchange $a$ and $b$ \;
			}
			$c \leftarrow a + b$ \;
			$z \leftarrow c - a$ \;
			$d \leftarrow b - z$ \;
			{\bf return} $(c,d)$ \;
			\caption{\texttt{FastTwoSum}} % Algorithm name
			\label{alg:fastTwoSum}   % optional label to refer to
		\end{algorithm}
	\end{minipage}
\end{center}

% Numbered question, with an optional title
\begin{question}[\itshape (with optional title)]
	In congue risus leo, in gravida enim viverra id. Donec eros mauris, bibendum vel dui at, tempor commodo augue. In vel lobortis lacus. Nam ornare ullamcorper mauris vel molestie. Maecenas vehicula ornare turpis, vitae fringilla orci consectetur vel. Nam pulvinar justo nec neque egestas tristique. Donec ac dolor at libero congue varius sed vitae lectus. Donec et tristique nulla, sit amet scelerisque orci. Maecenas a vestibulum lectus, vitae gravida nulla. Proin eget volutpat orci. Morbi eu aliquet turpis. Vivamus molestie urna quis tempor tristique. Proin hendrerit sem nec tempor sollicitudin.
\end{question}


% File contents
\begin{file}[hello.py]
\begin{lstlisting}[language=Python]
#! /usr/bin/python

import sys
sys.stdout.write("Hello World!\n")
\end{lstlisting}
\end{file}


% Command-line "screenshot"
\begin{commandline}
	\begin{verbatim}
		$ chmod +x hello.py
		$ ./hello.py

		Hello World!
	\end{verbatim}
\end{commandline}

% Warning text, with a custom title
\begin{warn}[Notice:]
  In congue risus leo, in gravida enim viverra id. Donec eros mauris, bibendum vel dui at, tempor commodo augue. In vel lobortis lacus. Nam ornare ullamcorper mauris vel molestie. Maecenas vehicula ornare turpis, vitae fringilla orci consectetur vel. Nam pulvinar justo nec neque egestas tristique. Donec ac dolor at libero congue varius sed vitae lectus. Donec et tristique nulla, sit amet scelerisque orci. Maecenas a vestibulum lectus, vitae gravida nulla. Proin eget volutpat orci. Morbi eu aliquet turpis. Vivamus molestie urna quis tempor tristique. Proin hendrerit sem nec tempor sollicitudin.
\end{warn}

%----------------------------------------------------------------------------------------

\bibliography{proyecto_visualizacion}{}
\bibliographystyle{acm}

\end{document}

